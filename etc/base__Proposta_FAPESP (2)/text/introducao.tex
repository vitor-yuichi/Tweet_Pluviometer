Os alagamentos são fenômenos cada vez mais frequentes em regiões urbanas devido ao aumento da população e o crescimento desordenado do processo de urbanização. Em São Paulo, os alagamentos são recorrentes desde os primórdios de sua ocupação, a estrutura urbana aliado às características dos rios existentes auxiliam na deflagração destes fenômenos \cite{hirata2013mapeamento}. Segundo \cite{santos2013impactos}, estima-se que os efeitos macroeconômicos dos alagamentos são de 172.3 milhões de reais por ano, afetando setores logísticos e industriais.
\par Diante dos impactos materiais, econômicos e humanos causados pelos alagamentos, é necessário medidas que visem a mitigação e antecipação deste fenômeno. Estas medidas estão associadas a uma infinidade de maneiras como os sistemas de alertas, essencial para que a comunidade seja alertada com antecedência de fenômenos naturais intensos e, desta forma, minimizar e previnir possíveis danos materiais e humanos \cite{kobiyama2006prevenccao}. 
\par Nesse ínterim, algumas pesquisas como de \cite{horita2015development} e  \cite{hirata2013mapeamento}, demonstram que a utilização de redes sociais que contém informações geográficas voluntárias podem ser um instrumento efetivo para o desenvolvimento de sistemas monitoramento e alertas das possíveis ocorrências de alagamento. A finalidade deste projeto é definir qual é o algoritmo de aprendizado de máquina que possui maior precisão com relação à previsão de alagamentos sobre determinado conjunto atributos. 
