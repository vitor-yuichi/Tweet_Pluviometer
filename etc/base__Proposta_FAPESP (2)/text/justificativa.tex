O  desenvolvimento  acelerado  de  São  Paulo  culminou  na  urbanização  descontrolada causando diversas consequências na região.   A impermeabilização do solo,  a drenagem urbana deficitária e a topografia favorável ao acúmulo de água,  são reflexos desta expansão desordenada. A deflagração de alagamentos causaram diversas perdas diretas e indiretas para o PIB de São Paulo, alcançando média de 172 milhões de reais em prejuízos econômicos por ano em algumas regiões do estado. Diante dos danos humanos, materiais e econômicos que os alagamentos vêm causando ao longo das décadas, é necessário medidas mitigadoras e o desenvolvimento de sistemas de alertas que possam antecipar a possível deflagração do fenômeno hidrológico. 
\par Além disso, em virtude da ascensão meteórica da tecnologia e da influência das redes sociais, o projeto visa o desenvolvimento da computação aplicada aos fenômenos hidrológicos de alagamentos, utilizando-se conceitos de ciência de dados e \textit{Machine Learning}.