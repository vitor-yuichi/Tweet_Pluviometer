%\hypertarget{estilo:resumo}{}


\begin{resumo}
Flooding is an increasingly frequent phenomenon and has been causing economic losses and even deaths in several cities worldwide, such as in the city of S�o Paulo, Brazil - the case study of this paper. Research shows that the use of voluntary geographic information is a helpful complementary tool for monitor and analyze a myriad of hydrological and meteorological events. Also, due to the technology that currently densely permeates the daily lives of the population, this project contemplates the development of computational methods with the support of data extracted from social networks in order to predict flooding phenomena.
\end{resumo}

%Classifica��o de imagens de Sensoriamento Remoto � uma das mais importantes aplica��es de Reconhecimento de Padr�es em estudos ambientais. A import�ncia de se obter resultados de classifica��o cada vez mais precisos motiva progressivamente o desenvolvimento e o aprimoramento das t�cnicas de classifica��o de imagens. 