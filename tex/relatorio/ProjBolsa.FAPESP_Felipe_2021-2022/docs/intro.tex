
\chapter{INTRODUCTION}\label{intro}

%Abertura e motiva��o...
Flooding is a frequent phenomenon in urban regions due to population growth and the
characteristics of the urbanization process, causing an increase of impervious surfaces
and poor drainage. This hydrological phenomenon is among the natural hazard associated
with the most significant impact in the world. Over the years, there has been an increase
in its global incidence. Global factors, such as Global warming, and local ones, such
as the lack of urban planning \cite{tingsanchali2012urban} are some of the variables causing this
increase.


In the city of S�o Paulo, Brazil, flooding has been recurrent since the beginning of its occupation. The urban structure combined with the hydrographic and morphological
characteristics helps trigger this phenomenon \cite{hirata2013mapeamento}. \citen{santos2013impactos}
estimated that the macroeconomic effects of flooding are 172.3 million reais per year. Based on the impacts mentioned, there is a growing trend in the literature incorporating tools such as the social network to predict flooding.

In this context, some researchers, such as those presented by \citen{horita2015development} and \citen{hirata2013mapeamento}, demonstrate that the use of social networks provided with voluntary geographic information can be used as an effective instrument in the development of flood monitoring and warning systems.

Similarly, \citen{de2015geographic} demonstrates the potential of spatiotemporal data obtained through the social network Twitter for disaster management. In the cited search, posts are filtered through keywords and then binary sorted among those ``related'' or ``not related'' to the event of interest, which in turn are aligned to a time series of outbreaks of flooding in a given region. The results obtained indicate the statistical relevance of the information pointed out by the Tweets in relation to the flooded regions. In a similar way \citen{vitorbrasil1} also demonstrates that the words used in the Twitter social network linked to the meteorological context on days of flooding are greater than in relation to days that do not occur.

In time, it is worth highlighting the recent rise in the use of Artificial Intelligence techniques, mainly Machine Learning methods, in the management and decision-making in the face of disaster events. In \citen{sit2019identifying} natural language processing algorithms are used for semantic identification of Tweets related to the context of disasters. Machine Learning is also incorporated into the analysis of multiple physical parameters useful in flood prediction, as shown by \citen{mosavi2018flood}.

Thus, in the context of the problem involving flooding events and the potential offered by Machine Learning methods, this research project aims to build an algorithm capable of predicting the occurrence of flooding through information automatically extracted from the social network. Twitter. Data obtained by meteorological radar, rain gauge and the flooding database are components that integrate the proposed algorithm.



