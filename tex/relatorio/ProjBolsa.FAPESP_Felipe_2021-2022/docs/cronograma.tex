\chapter{CONCLUSION AND PERSPECTIVES} \label{Cron}

The first project work indicates a statistical relation between the tweets associated with the meteorological and hydrological context with flooding. Due to the myriad of factors linked to the outbreak of this hydrological phenomenon, other supplementary monitoring mechanisms are needed. In short, Twitter can be a powerful complementary tool. By joining this social network with other monitoring stations, sophisticated monitoring and alerting systems can be developed.

In general, in the second project it can be observed that the value of the attributes with the floods have moderate and strong correlations. Demonstrating that there is potential to submit data to classification algorithms.

In possession of primary analysis, the next step is to prepare the data for Machine Learning algorithms, optimize data quality and detect outliers. 


%\newpage{}
