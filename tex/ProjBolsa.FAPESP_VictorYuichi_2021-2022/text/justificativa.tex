%%%%%%%%%%%%%%%%%%%%%%%%%%%%%%%%%%%
\chapter{Justificativa e Relev�ncia}\label{secReleva}

%RGN: Vamos padronizar alguns termos - Aprendizado de M�quina � um deles.
%RGN: Verificar trechos j� usados em outras partes e mudar um pouco mais o texto

O  desenvolvimento  acelerado  de  S�o  Paulo  culminou  na  urbaniza��o  descontrolada causando diversas consequ�ncias na regi�o.   A impermeabiliza��o do solo,  a drenagem urbana deficit�ria e a topografia favor�vel ao ac�mulo de �gua,  s�o reflexos desta expans�o desordenada. A deflagra��o de alagamentos causaram diversas perdas diretas e indiretas nos mais diversos setores da economia no estado. Em vista da consequ�ncias que os alagamentos v�m causando ao longo das d�cadas, � necess�rio medidas mitigadoras e o desenvolvimento de tecnologias que possam antecipar a poss�vel deflagra��o do fen�meno hidrol�gico. 


Al�m disso, em virtude da ascens�o mete�rica da tecnologia e da influ�ncia das redes sociais, o projeto visa o desenvolvimento da computa��o aplicada aos fen�menos hidrol�gicos de alagamentos, utilizando-se conceitos de ci�ncia de dados e Aprendizado de M�quina.


