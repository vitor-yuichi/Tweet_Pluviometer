%%%%%%%%%%%%%%%%%%%%%%%%%%%%%%%%%%%
\chapter{Objetivos}\label{secObjetivos}

%RGN: Acredito que seja necess�rio omitir esse vinculo com o cemaden, pois n�o existe discuss�o anterior que deixaria isso claro. Ainda, acho que essa info n�o seja relevante para a Fapesp.

A finalidade deste projeto � o desenvolvimento de um modelo de Aprendizado de M�quina para a predi��o de alagamentos, utilizando-se dos diferentes algoritmos propostos pela pesquisa. Neste projeto ser� utilizado s�ries temporais da rede social Twitter, a base de dados de alagamentos em S�o Paulo, e dados pluviom�tricos e meteorol�gicos.

%RGN: incluir mais objetivos especificos: publicar artigo, construir e disponibilizar um banco de dados sobre alagamentos e dados de tweet, disponibilizar os c�digos em reposit�rio publico, [veja o exemplo do Felipe]
Objetivos espec�ficos: 
\begin{enumerate}
    \item Definir qual � o algoritmo mais preciso para relacionar os dados meteorol�gicos, pluviom�tricos e frequ�ncia de tweets, para o modelo de classifica��o de alagamentos em dias de "ocorr�ncia" e "n�o-ocorr�ncia".
    
    \item Estabelecer o atributo com maior relev�ncia estat�stica para predi��o de alagamentos dentre os propostos, combinando-os de forma conjunta com os algoritmos. 
    
    \item Publica��o dos resultados em congressos, eventos e peri�dicos cient�ficos.
    
    \item Estruturar e disponibilizar as bases de dados e c�digos em um reposit�rio p�blico. 
\end{enumerate}


