\documentclass[PublicacaoArtigoOuRelatorio,english,portuguese,LogoINPE]{tdiinpe}


%%%%%%%%%%%%%%%%%%%%%%%%%%%%%%%%%%%%%%%%%%%%%
%%% Pacotes j� previamente carregados:      %
%%%%%%%%%%%%%%%%%%%%%%%%%%%%%%%%%%%%%%%%%%%%%%%%%%%%%%%%%%%%%%%%%%%%%%%%
%%% ifthen,calc,graphicx,color,inputenc,babel,hyphenat,array,setspace, %
%%% bigdelim,multirow,supertabular,tabularx,longtable,lastpage,lscape, %
%%% rotate,caption2,amsmath,amssymb,amsthm,subfigure,tocloft,makeidx,  %
%%% eso-pic,calligra,hyperref,ae,fontenc                               %
%%%%%%%%%%%%%%%%%%%%%%%%%%%%%%%%%%%%%%%%%%%%%%%%%%%%%%%%%%%%%%%%%%%%%%%%

\usepackage{dsfont}
\usepackage{comment}
\usepackage{setspace}
%ADICOINAIS%%%%%%%%%%%%%%%%%%%%%%%%%%%%%%%%%%%%%%%%%%%%%
\usepackage{rotating}
\usepackage{amssymb}
\usepackage{bm}
\usepackage{color}
\usepackage{colortbl}
\usepackage{mathrsfs}
\usepackage{enumerate}
\usepackage[section]{placeins}
\usepackage{siunitx}
\DeclareMathOperator*{\argmax}{arg\,max}
\DeclareMathOperator*{\argmin}{arg\,min}
\DeclareMathOperator*{\maxi}{max}
\DeclareMathOperator*{\mini}{min}
\DeclareMathOperator*{\sgn}{sgn}

%\usepackage{natbib}

\usepackage{pdfpages}

%\renewcommand{\labelenumi}{\arabic{enumi}.}
\renewcommand{\thefootnote}{(\arabic{footnote})}

%\makeatletter
%\newcommand{\thickhline}{\noalign {\ifnum 0=`}\fi \hrule height 1pt \futurelet \reserved@a \@xhline}
%\newcolumntype{"}{@{\hskip\tabcolsep\vrule width 1pt\hskip\tabcolsep}}
%\makeatother
%%%%%%%%%%%%%%%%%%%%%%%%%%%%%%%%%%%%%%%%%%%%%%%%%%%%%%%%



%%%%%%%%%%%%%%%%%%%CAPA%%%%%%%%%%%%%%%%%%%%%%%%%%%%%%%%
%\serieinpe{INPE-NNNNN-TDI/NNNN} %% n�o mais usado

%\titulo{CARACTERIZA��O DAS DIN�MICAS TEMPORAIS DO USO E COBERTURA DO SOLO EM �REAS DO PANTANAL AFETADAS PELAS QUEIMADAS RECENTESNO ANO DE 2020}
\titulo{FLOODING FORECAST VIA INTEGRATION
OF TWEETS, WEATHER DATA AND
MACHINE LEARNING ALGORITHMS}

\title{}
\author{Student: Vitor Yuichi Hossaki \\ Supervisor: Prof. Dr. Leonardo Bacelar Lima Santos}
\descriccao{Report to CNPq/Cemaden Scientific Initiation Scholarship Program}
\repositorio{}
\tipoDaPublicacao{}
\IBI{}

\date{Descember of 2021}

%%%%%%%%%%%%%%%%%%%%%%%%%%%VERSO DA CAPA%%%%%%%%%%%%%%%%%%%%%%%%%%%%%%%%%%%%%%%%%%%%%%%
%\tituloverso{\vspace{-0.9cm}\textbf{\PublicadoPor:}}
%\descriccaoverso{%Instituto Nacional de Pesquisas Espaciais - INPE\\
%%Gabinete do Diretor (GB)\\
%%Servi�o de Informa��o e Documenta��o (SID)\\
%%Caixa Postal 515 - CEP 12.245-970\\
%%S�o Jos� dos Campos - SP - Brasil\\
%%Tel.:(012) 3945-6923/6921\\
%%Fax: (012) 3945-6919\\
%%E-mail: {\url{pubtc@sid.inpe.br}}
%}
%
%% CEPPII de 09/12/2011 a 08/12/2013:
%\descriccaoversoA{\textbf{\ConselhoDeEditoracao:}\\
%\textbf{\Presidente:}\\
%Marciana Leite Ribeiro - Servi�o de Informa��o e Documenta��o (SID)\\
%\textbf{\Membros:}\\
%Dr. Antonio Fernando Bertachini de Almeida Prado - Coordena��o Engenharia e Tecnologia Espacial (ETE)\\
%Dr� Inez Staciarini Batista - Coordena��o Ci�ncias Espaciais e Atmosf�ricas (CEA)\\
%Dr. Gerald Jean Francis Banon - Coordena��o Observa��o da Terra (OBT)\\
%Dr. Germano de Souza Kienbaum - Centro de Tecnologias Especiais (CTE)\\
%Dr. Manoel Alonso Gan - Centro de Previs�o de Tempo e Estudos Clim�ticos (CPT)\\
%Dr� Maria do Carmo de Andrade Nono - Conselho de P�s-Gradua��o\\
%Dr. Pl�nio Carlos Alval� - Centro de Ci�ncia do Sistema Terrestre (CST)\\
%\textbf{\BibliotecaDigital:}\\
%Dr. Gerald Jean Francis Banon - Coordena��o de Observa��o da Terra (OBT)\\
%%Jefferson Andrade Ancelmo - Servi�o de Informa��o e Documenta��o (SID)\\
%%Simone A. Del-Ducca Barbedo - Servi�o de Informa��o e Documenta��o (SID)\\
%%Deicy Farabello - Centro de Previs�o de Tempo  e Estudos Clim�ticos (CPT)\\
%\textbf{\RevisaoNormalizacaoDocumentaria:}\\
%Marciana Leite Ribeiro - Servi�o de Informa��o e Documenta��o (SID) \\
%%Maril�cia Santos Melo Cid - Servi�o de Informa��o e Documenta��o (SID)\\
%Yolanda Ribeiro da Silva Souza - Servi�o de Informa��o e Documenta��o (SID)\\
%\textbf{\EditoracaoEletronica:}\\
%Marcelo de Castro Pazos - Servi�o de Informa��o e Documenta��o (SID)\\
%}

%%%%%%%%%%%%%%%%%%%FOLHA DE ROSTO

%%%%%%%%%%%%%%%%FICHA CATALOGR�FICA
%%% N�O PREENCHER - SER� PREENCHIDO PELO SID
%
%\cutterFICHAC{Cutter}
%\autorUltimoNomeFICHAC{Negri, Rog�rio Galante}
%\autorAbreviadoFICHAC {RGN}
%\tituloFICHAC{INVESTIGA��O E DESENVOLVIMENTO DE ALGORITMOS PARA DETEC��O DE MUDAN�A EM IMAGENS DE SENSORIAMENTO REMOTO}
%\instituicaosigla{}
%\instituicaocidade{S�o Jos� dos Campos}
%\paginasFICHAC{\pageref{} + \pageref{}}
%\palavraschaveFICHAC{1.~Palavra chave. 2.~Palavra chave 3.~Palavra chave. 4.~Palavra chave. 5.~Palavra chave  I.~\mbox{T�tulo}.}
%\numeroCDUFICHAC{000.000} %% n�mero do CDU 
%
%% Nota da ficha (para TD)
%\tipoTD{Projeto de Pesquisa}
%\cursoFA{}
%\instituicaoDefesa{}
%\anoDefesa{2013}
%\nomeAtributoOrientadorFICHAC{}
%\valorAtributoOrientadorFICHAC{}
%
%%%%%%%%%%%%%%%%FOLHA DE APROVA�AO PELA BANCA EXAMINADORA
%\tituloFA{\textbf{ATEN��O! A FOLHA DE APROVA��O SER� INCLUIDA POSTERIORMENTE.}}
%%\cursoFA{\textbf{}}
%\candidatoOUcandidataFA{}
%\dataAprovacaoFA{}
%\membroA{}{}{}
%\membroB{}{}{}
%\membroC{}{}{}
%\membroD{}{}{}
%\membroE{}{}{}
%\membroF{}{}{}
%\membroG{}{}{}
%\ifpdf

%%%%%%%%%%%%%%%N�VEL DE COMPRESS�O {0 -- 9}
%\pdfcompresslevel 9
%\fi
%%%% define em 80% a largura das figuras %%%
%\newlength{\mylenfig} 
%\setlength{\mylenfig}{0.8\textwidth}
%%%%%%%%%%%%%%%%%%%%%%%%%%%%%%%%%%%%%%%%%%%%
%
%%%%%%%%%%%%%%%COMANDOS PESSOAIS
%\newcommand{\vetor}[1]{\mathit{\mathbf{#1}}}


\makeindex

\begin{document}

\maketitle

%%\hypertarget{estilo:resumo}{}


\begin{resumo}
Flooding is an increasingly frequent phenomenon and has been causing economic losses and even deaths in several cities worldwide, such as in the city of S�o Paulo, Brazil - the case study of this paper. Research shows that the use of voluntary geographic information is a helpful complementary tool for monitor and analyze a myriad of hydrological and meteorological events. Also, due to the technology that currently densely permeates the daily lives of the population, this project contemplates the development of computational methods with the support of data extracted from social networks in order to predict flooding phenomena.
\end{resumo}

%Classifica��o de imagens de Sensoriamento Remoto � uma das mais importantes aplica��es de Reconhecimento de Padr�es em estudos ambientais. A import�ncia de se obter resultados de classifica��o cada vez mais precisos motiva progressivamente o desenvolvimento e o aprimoramento das t�cnicas de classifica��o de imagens. 

\includeSumario
\inicioIntroducao

\doublespacing

%%%%%%%%%%%%%%%%%%%%%%%%%%%%%%%%%%%
\chapter{Introdu��o}
Os alagamentos s�o fen�menos cada vez mais frequentes em regi�es urbanas devido ao aumento da popula��o e o crescimento desordenado do processo de urbaniza��o. Em S�o Paulo, os alagamentos s�o recorrentes desde os prim�rdios de sua ocupa��o, a estrutura urbana aliado �s caracter�sticas dos rios existentes auxiliam na deflagra��o destes fen�menos \cite{hirata2013mapeamento}. Segundo \cite{santos2013impactos}, estima-se que os efeitos macroecon�micos dos alagamentos s�o de 172.3 milh�es de reais por ano, afetando setores log�sticos e industriais.

Diante dos impactos materiais, econ�micos e humanos causados pelos alagamentos, � necess�rio medidas que visem a mitiga��o e antecipa��o deste fen�meno. Estas medidas est�o associadas a uma infinidade de maneiras como os sistemas de alertas, essencial para que a comunidade seja alertada com anteced�ncia de fen�menos naturais intensos e, desta forma, minimizar e previnir poss�veis danos materiais e humanos \cite{kobiyama2006prevenccao}. 

Nesse �nterim, algumas pesquisas como de \cite{horita2015development} e  \cite{hirata2013mapeamento}, demonstram que a utiliza��o de redes sociais que cont�m informa��es geogr�ficas volunt�rias, podem ser um instrumento efetivo para o desenvolvimento de sistemas monitoramento e alertas das poss�veis ocorr�ncias de alagamento.

O autor \cite{de2015geographic} demonstra a potencialidade da rede social Twitter para a gest�o de desastres, combinando-se dados hist�rico espa�o temporais desta rede. As postagens s�o filtradas atrav�s de palavras chaves e classificadas binariamente como relacionado ou n�o-relacionada, aliando-se � s�rie temporal de deflagra��es dos alagamentos em determinada regi�o. Neste contexto, os dados s�o analisados estatisticamente atrav�s de GLMs (modelos lineares generalizados). Por fim, o trabalho indica a relev�ncia estat�sticas dos Tweets relacionados, pr�ximos �s regi�es de alagamento ($\leq10km$ do local de ocorr�ncia). 

Al�m disso, � verificado uma ascens�o na literatura quanto ao emprego de intelig�ncia artificial, principalmente o Machine Learning, para a gest�o e tomada de decis�o em desastres. Na pesquisa de \cite{sit2019identifying}, s�o utilizados algoritmos de processamento de linguagem natural para identificar a sem�ntica de Tweets relacionados ao contexto de desastres. O aprendizado de m�quina tamb�m � incorporado para analisar m�ltiplos par�metros f�sicos para a previs�o de alagamentos \cite{mosavi2018flood}.

Em face do crescimento da utiliza��o de intelig�ncia artificial e das consequ�ncias dos alagamentos, a finalidade deste projeto � definir qual � o algoritmo de aprendizado de m�quina que possui maior precis�o com rela��o � previs�o de alagamentos, utilizando-se um conjunto de atributos aliado � rede social Twitter. 

A estrutura geral da proposta � dividida e enunciada a seguir; na Se��o 2 � esclarecido qual � o objetivo principal e espec�fico do trabalho; na Se��o 3 a relev�ncia e import�ncia do tema proposto;  sobre as quest�es centrais abordadas no trabalho e conceitos explicado de maneira mais espec�fica s�o encontradas na Se��o 4; a Se��o 5 � de destinada � delimita��o da �rea de estudo e a disponibilidade e origem dos dados e a metodologia prevista;  por fim na se��o 6 � encontrada o plano de atividade e cronograma. 


%%%%%%%%%%%%%%%%%%%%%%%%%%%%%%%%%%%
\chapter{Objetivos}

A finalidade deste projeto � alinhar-se com os objetivos do Centro Nacional de Alerta de Desastres Naturais (CEMADEN) no desenvolvimento de meios para prever alagamentos, partindo-se da hip�tese que � poss�vel utilizar algoritmos de aprendizado de m�quina para detectar alagamentos, e desta forma, emitindo-se alertas.


Objetivos espec�ficos: 
\begin{enumerate}
    \item Definir qual � o algoritmo mais preciso para relacionar os dados metereol�gicos, pluviom�tricos e frequ�ncia de tweets para o modelo de classifica��o de dias de alagamentos e n�o-alagamentos. 
\end{enumerate}



%%%%%%%%%%%%%%%%%%%%%%%%%%%%%%%%%%%
\chapter{Justificativa e Relev�ncia}

O  desenvolvimento  acelerado  de  S�o  Paulo  culminou  na  urbaniza��o  descontrolada causando diversas consequ�ncias na regi�o.   A impermeabiliza��o do solo,  a drenagem urbana deficit�ria e a topografia favor�vel ao ac�mulo de �gua,  s�o reflexos desta expans�o desordenada. A deflagra��o de alagamentos causaram diversas perdas diretas e indiretas para o PIB de S�o Paulo, alcan�ando m�dia de 172 milh�es de reais em preju�zos econ�micos por ano em algumas regi�es do estado. Diante dos danos humanos, materiais e econ�micos que os alagamentos v�m causando ao longo das d�cadas, � necess�rio medidas mitigadoras e o desenvolvimento de sistemas de alertas que possam antecipar a poss�vel deflagra��o do fen�meno hidrol�gico. 


Al�m disso, em virtude da ascens�o mete�rica da tecnologia e da influ�ncia das redes sociais, o projeto visa o desenvolvimento da computa��o aplicada aos fen�menos hidrol�gicos de alagamentos, utilizando-se conceitos de ci�ncia de dados e \textit{Machine Learning}.



%%%%%%%%%%%%%%%%%%%%%%%%%%%%%%%%%%%
\chapter{Fundamenta��o Te�rica}\label{secFundamentos}

%%%%%%%%%%%%%%%%%%
\section{Alagamentos}
Os alagamentos s�o fen�menos associados ao ac�mulo de �gua em determinado local, favorecidos pela microdrenagem e macrodrenagem insuficientes. A aus�ncia de planejamento urbano e a r�pida modifica��o do espa�o culmina na impermeabiliza��o do solo, contribuindo para diminui��o do tempo concentra��o e o aumento do volume de escoamento superficial, amplificando-se assim, o pico da vaz�o e consequentemente saturando a drenagem pluvial do local \cite{hansmann2013descriccao}. 

A topografia e a eleva��o do local tamb�m s�o fatores preponderantes para a ocorr�ncia de alagamentos, ou seja, verifica-se que os lugares com maior frequ�ncia de alagamentos tem caracter�sticas morfom�tricas planas, depress�es ou fundos de vales, dificultanto o processo de escoamento superficial do local \cite{braga2016alagamentos}.

Fatores como o descarte inadequado de res�duos s�lidos, podem causar obstru��o dos sistema de drenagem do local, isto ocorre em decorr�ncia da aus�ncia de educa��o ambiental da popula��o. 

Os alagamentos s�o fen�menos complexos, uma vez que a sua causa est�  interrelacionada a uma gama par�metros como o clima que incluem precipita��o forte ou persistente, falhas estruturais urbanas, sistemas de drenagem inadequados, bacias hidrogr�ficas, proximidades de corpos aqu�ticos, uso e ocupa��o inapropriado do solo e entre outros \cite{doocy2013human}. 


%%%%%%%%%%%%%%%%%%
\section{A utiliza��o de redes sociais para o monitoramento de eventos}
O desenvolvimento da sociedade na esfera tecnol�gica permitiu a ascens�o mete�rica das redes sociais e suas funcionalidades. A quantidade massiva de dados gerados das redes sociais consolidam a intera��o do universo virtual com o mundo concreto, onde usu�rios expressam suas percep��es e emo��es acerca dos eventos circundantes \cite{naaman2011geographic}. A atividade das redes sociais e sua heterogeneidade espacial demonstra a potencialidade para o monitoramento de eventos metereol�gicos como a precipita��o \cite{de2021effect}. 

Atrav�s das plataformas de m�dia social, uma �nica postagem pode ser vista por milhares de usu�rios simultaneamente, al�m disso algumas plataformas utilizam-se de georreferenciamento que permite a visualiza��o n�o s� da postagem como tamb�m a localiza��o do usu�rio com seu dispositivo m�vel. Redes sociais (Twitter) ou aplicativos como Open Street Map que permitem a tecnologia de georreferenciamento s�o denominadas informa��es geogr�ficas volunt�rias, o trabalho de \cite{horita2015development}, integra estas plataformas para o gerenciamento de risco dos alagamentos. 

A utiliza��o das redes sociais apresentam uma crescente tend�ncia na sua incorpora��o em pesquisas para o monitoramento e an�lise de uma infinidade de eventos. Segundo \cite{de2015geographic}, a utiliza��o de informa��es geogr�ficas volunt�rias, principalmente a rede a social Twitter, s�o componentes fundamentais para a maior consci�ncia dos eventos ocorrentes ou seja, consolida-se a percep��o dos elementos no ambiente e possibilita maior compreens�o das poss�veis consequ�ncias. 
 
 
%%%%%%%%%%%%%%%%%%
\section{Classifica��o}
O Aprendizado de M�quina � cada vez mais empregado pelos pesquisadores na �rea de desastres de naturais, alguns autores utilizam esta ferramenta para analisar a sem�ntica atrelada das postagens de rede social, e assim, aprimorar os resultados da classifica��o de determinada ocorr�ncia \cite{de2015geographic, deparday2019machine}. 

Esta tecnologia pode ser definida como um conjunto m�todos computacionais para aprimorar performance ou realizar predi��es acuradas. A classifica��o � um dos m�todos computacionais amplamente utilizados para categoriza��o de cada item em uma s�rie de dados. Matematicamente, a classifica��o � descrita por uma fun��o \(F: \mathcal{X} \rightarrow \mathcal{Y}\) que associa elementos no conjunto de atributos \(\mathcal{X}\) a uma classe de \(\Omega=\{\omega_1, \omega_2,...,\omega_n\}\), com \(n \in \mathbb{N}^*\), e partindo-se de um indicador de classe \(\mathcal{Y}=\{1,2,...,n\}\), portanto, quando \(x \in \mathcal{X}\) e \(y \in \mathcal{Y}\), a fun��o \(y=F(x)\) indica que x pertence � \(\Omega_y\).

Os modelos de aprendizagem supervisionada, a fun��o \(F\) utiliza-se das informa��es do conjunto de treinamento representado pela equa��o \(\mathcal{D}=\{(x_j,\omega_j \in \times \Omega : i=1,...m; \ j=1,...,c \}\), no qual \(m \) � a quantidade de dados no treinamento.


Atualmente os principais algoritmos empregados para classifica��o s�o o \textit{Suport Vector Machine} (SVMs) e \textit{Random Forest} (RF) \cite{mohri2018foundations}. \textcolor{red}{e as redes neurais? (MLP?)}



%-----------------
\subsection{SVM}
O m�todo SVM realiza a distin��o entre amostras de treinamento partindo-se de um hiperplano que possui maior abrang�ncia de separa��o, mapeando o padr�o de vetores para um espa�o de alta dimens�o, determinando-se o hiperplano mais adequado para separa��o de dados. Este algoritmo � utilizado por diversos autores devido � alta acur�cia para problemas de classifica��o bin�ria \cite{lian2006multi} . 

O hiperplano corresponde ao lugar geom�trico nos quais a fun��o \(f(x)=\langle w,x \rangle+b\) � nula. A vari�vel \(w\) � o vetor ortogonal ao hiperplano e \(b\) a dist�ncia entre a fun��o e a origem do espa�o de atributos. 

Para se encontrar o hiperplano mais adequado para separa��o entre as classes, � necess�rio a resolu��o do problema de otimiza��o \cite{theodoridis2010introduction} representado por:
\(
    max_\gamma (\sum^m_{i=1} \gamma_i-\frac{1}{2}\sum^m_{i=1}\sum^m_{j=1}\gamma_i \gamma_j y_i y_j \langle x_i,x_j \rangle), \ 
    \begin{cases}
0 \leq \gamma_i \leq \mathcal{C}, i=1,...,m \\
    \sum^m_{i=1} \gamma_i y_i=0
\end{cases}
\), 
a vari�vel \(\mathcal{C}\) � o par�metro utilizado para regulariza��o para ajustar o hiperplano e \(\gamma_i\) s�o os multiplicadores de Lagrange. 


A defini��o dos par�metros \(w\) e \(b\) que comp�em o hiperplano s�o dadas por: 
\(
    w=\sum_{\forall x_i \in SV}y_i \gamma_i x_i, \  
    b=\frac{1}{\#SV}(\sum_{x_i \in SV} y_i+ \sum_{x_i \in SV} \cdot \sum_{x_j \in SV} \gamma_i \gamma_j y_i y_j \langle x_i,x_j \rangle) 
\), \(SV\) � um subconjunto das amostrar de treinamento \(\mathcal{D}\), nos quais os elementos s�o os vetores suporte. Por fim indica��o da classe pertencente do vetor analisado � dado pelo sinal da fun��o discriminante \(f(x)\) \cite{maselli2019integraccao}.


%-----------------
\subsection{RF}
A classifica��o atrav�s do algoritmo Floresta Aleat�ria vem sendo amplamente utilizada literatura para avalia��o e mapeamento dos padr�es de eventos hidrol�gicos. Pesquisa como de \cite{zhu2021flood} e \cite{liu2020random} demonstram a potencialidade do algoritmo para avaliar a resili�ncia e os padr�es espaciais dos alagamentos. 


Este modelo � um algoritmo de classifica��o que representa um conjunto de �rvores de decis�o, que combina a sa�da destas diversas �rvores atribuindo-se uma classe ao conjunto de dados. Segundo \cite{breiman2001random}, a Floresta Aleat�ria consiste em uma cole��o de classificadores em forma de �rvore descritos por \(\{h(x,\theta_k), k=1,..\}\) onde \(\theta_k\) s�o independentes e em cada �rvore � lan�ado um voto unit�rio para a classe mais popular para o input \(x\). 


%-----------------
\subsection{Redes Neurais}
Este algoritmo vem sendo empregado para emiss�o de alertas hidrol�gicos e mapeamentos de suscetibilidade em alguns autores como \cite{da2016utilizaccao} e \cite{pacheco2020mapeamento}, demonstrando efetividade e acur�cia elevada para os modelo de previss�o associados aos fen�menos hidrol�gicos. 


A t�cnica \textit{Multilayer Perceptron} demonstra resultados relevantes as mais diversas �reas da ci�ncia \cite{gardner1998artificial}. Este algoritmo consiste em um sistema interconectado de neur�nios, estes n�s s�o conectados entre si por um peso. Matematicamente as camadas de neur�nios de entrada e sa�da s�o vetores definidos como $i$ e $O$ respectivamente, e os pesos como uma matriz $W$. Portanto a sa�da da rede � dada por $O=f(IW_{io})$, ao final do processo uma fun��o determina se aquele n� ser� ativado na condi��o $f(x)\begin{cases} 
1 \ \ x>0 \\
0 \ \ otherwise
\end{cases}$. Assumindo que $T$ � o par�metro de sa�da para o vetor de treinamento, o algoritmo o calcula o erro associado atrav�s de $E(O)=T-O=T-f(IW_{io})$. Algumas t�cnicas visam a redu��o do erro atrav�s da atualiza��o dos pesos no processo representado matematicamente por $W_{io}(t+1)=W_{io}(t)+\alpha E_n$.



%%%%%%%%%%%%%%%%%%%%%%%%%%%%%%%%%%%
\chapter{Materiais e M�todos}


%%%%%%%%%%%%%%%%%%
\section{�rea de estudo e dados dispon�veis}
O estudo foi realizado na regi�o de S�o Paulo onde est� localizado a bacia hidrogr�fica do Rio Tamanduate� (Figura \ref{fig:area_estudo}). Esta bacia possui uma �rea de \(323km^2\) e se estende at� as bacia hidrogr�ficas do Rio Pinheiro, Rio Guai�, Rio Aricanduva e C�rrego de Tapuap�. Nesta regi�o, foi analisado a partir de um pluvi�metro um raio espacial de \(2000m\) que abrange as regi�es de alagamentos, tweets georrefenciados e a c�lula de radar.
\begin{figure}[H]
    \centering
    \includegraphics[scale=0.4]{imagens/ic_att2.png}
    \caption{�rea de estudo }
    \label{fig:area_estudo}
\end{figure}


Os dados da rede social Twitter foram extra�das atrav�s da API (\textit{Application Programming Interface}). Os dados pluviom�tricos s�o coletados do pluvi�metro 833A, pertencente ao Centro Nacional e Alertas e Desastres Naturais (CEMADEN), estes dados podem ser encontrados no pr�prio site da institui��o.  


A s�rie hist�rica de alagamentos na �rea de estudo, foram concebidas por um dos integrantes da pesquisa. Os dados metereol�gicos foram extra�dos por esta��es pertencentes ao CEMADEN, o equipamento est� localizado na cidade de S�o Roque - SP e atualmente est� em opera��o pelo Departamento de Controle do Espa�o A�reo (DECEA). Esse radar tem alcance de 250 km, cobrindo toda a regi�o metropolitana de S�o Paulo. O produto de radar usado para o CAPPI (Constant Altitude Plan Position Indicator) na altura de 3 km. Este produto possui uma resolu��o espacial de aproximadamente 1 km e uma resolu��o temporal de 10 minutos. Para a convers�o da refletividade (dBZ) em taxa de separa��o (mm / h) foi utilizado em rela��o a Marshall-Palmer \cite{marshall1948mc}) e a seguir os dados foram acumulados por dia. 

%%%%%%%%%%%%%%%%%%
\section{Ferramentas Computacionais}
A an�lise e aplica��o do projeto ser� realizada de maneira geral com a ferramenta \textit{Python}. Para a manipula��o, filtragem e tratamento dos dados ser� utilizada a biblioteca \textit{Pandas}, j� a an�lise gr�fica com \textit{Matplotlib} e \textit{Seaborn}. 


A aplica��o de testes estat�sticos na s�rie de dados ser� usado \textit{Scipy} e \textit{Numpy}, para os modelos de apredizagem supracitados, a biblioteca espec�fica para aprendizado de m�quina denominado \textit{Scikit-learn}. Por fim, algumas filtra��es no banco de dados de alagamentos ser� realizada com ferramentas de geoprocessamento do software \textit{QGIS}. 


%%%%%%%%%%%%%%%%%%
\section{Proposta de algoritmo para defini��o do modelo}
A concep��o inicial deste trabalho � analisar as s�ries temporais  dos alagamentos, tweets, pluvi�metro e radar, para definir quais s�o o melhor conjuntos de par�metros em dias de alagamentos, associando-se ao n�mero m�nimo necess�rio de tweets para emiss�o de um alerta. 


A s�rie temporal analisada compreende os tr�s primeiros meses do ano de 2019. Para a base de dados dos tweets, o processamento consiste no recorte temporal e filtra��o do tweets com base na lista de palavras associadas ao contexto metereol�gico e hidrol�gico. Esta lista de palavra basea-se no trabalho de \cite{de2021effect}. 


Com base na estrutura (Figura \ref{fig:metodologia}), ser� registrado em �nico arquivo, na mesma s�rie temporal, o n�mero de tweets filtrados, os valores de precipita��o do radar e o pluviom�tro, e se houve alagamentos no dias analisados. Este dados processados em um �nico arquivo, possibilitar�o a submiss�o nos modelos de aprendizados propostos, dividindo-se em base de dados para teste e treinamento.  
\begin{figure}[H]
    \centering
    \includegraphics[scale=0.4]{imagens/medotologia.png}
    \caption{Metodologia}
    \label{fig:metodologia}
\end{figure}


Como a classifica��o bin�ria consiste em dias de alagamento e n�o alagamento, a acur�cia ser� medida a partir da base de dados de teste. Ap�s o treinamento nos modelo SVM e Floresta Aleat�ria e Redes Neurais, ser�o analisadas a acur�cia atrav�s da valida��o cruzada e subsequentemente testes estat�sticos como ANOVA e coeficiente Kappa, determinando-se assim, o algoritmo que possui maior potencial para o desenvolvimento de um sistema de alerta com base nos dados dispon�veis. 



%\section{Cronograma}
%A pesquisa ser� realizada em 12 meses e ser� executada nos passos listado abaixo (Tabela 1) \\ 
%A - Revis�o sistem�tica em desastres associados � alagamentos e modelos de classifica��o; \\
%B - Estudo dos modelos de aprendizado de m�quina e aplica��o em Python; \\
%C - Processamento dos bancos de dados; \\
%D - An�lise explorat�ria dos dados processados; \\
%E - Submiss�o dos dados processados para treinamento nos modelos propostos; \\
%F - Classifica��o; \\  
%G - C�lculos estat�sticos e infer�ncias;\\
%H - Altera��es, ajustes e otimiza��es no modelo de melhor desempenho; \\
%I - An�lise e conclus�o dos resultados; \\
%J - Relat�rio final 
%\begin{table}[H]
%    \centering
%    \begin{tabular}{|l|l|l|l|l|l|l|l|l|l|l|l|l|l|}
%\hline
%M�s &  & 1� & 2� & 3� & 4� & 5� & 6� & 7� & 8� & 9� & 10� & 11� & 12� \\ \hline
%\multirow{10}{*}{\begin{turn}{90} Etapas \end{turn}} & A & $\bullet$ & $\bullet$ &  &  &  &  &  &  &  &  &  &  \\ \cline{2-14} 
% & B & $\bullet$ & $\bullet$ & $\bullet$ &  &  &  &  &  &  &  &  &  \\ \cline{2-14} 
% & C &  &  & $\bullet$ & $\bullet$ &  &  &  &  &  &  &  &  \\ \cline{2-14} 
% & D &  &  &  & $\bullet$ &  &  &  &  &  &  &  &  \\ \cline{2-14} 
% & E &  &  &  &  & $\bullet$ & $\bullet$ &  &  &  &  &  &  \\ \cline{2-14} 
% & F &  &  &  &  & $\bullet$ & $\bullet$ &  &  &  &  &  &  \\ \cline{2-14} 
% & G &  &  &  &  &  &  & $\bullet$ & $\bullet$ &  &  &  &  \\ \cline{2-14} 
% & H &  &  &  &  &  &  &  & $\bullet$ & $\bullet$ &  &  &  \\ \cline{2-14} 
% & I &  &  &  &  &  &  &  &  &  & $\bullet$ & $\bullet$ &  \\ \cline{2-14} 
% & J &  &  &  &  &  &  &  &  &  &  & $\bullet$ & $\bullet$ \\ \hline
%\end{tabular}
%    \caption{Cronograma}
%    \label{tab:my_label}
%\end{table}

%%%%%%%%%%%%%%%%%%%%%%%%%%%%%%%%%%%
\chapter{Plano de Atividades e Cronograma}



%\section{Cronograma}

A pesquisa ser� realizada em 12 meses e ser� executada nos passos listado abaixo (Tabela 1).
\begin{itemize}
\item A - Revis�o sistem�tica em desastres associados � alagamentos e modelos de classifica��o; 
\item B - Estudo dos modelos de aprendizado de m�quina e aplica��o em Python; 
\item C - Processamento dos bancos de dados; 
\item D - An�lise explorat�ria dos dados processados; 
\item E - Submiss�o dos dados processados para treinamento nos modelos propostos; 
\item F - Classifica��o; 
\item G - C�lculos estat�sticos e infer�ncias;
\item H - Altera��es, ajustes e otimiza��es no modelo de melhor desempenho; 
\item I - An�lise e conclus�o dos resultados; 
\item J - Relat�rio final 
\end{itemize}

\begin{table}[H]
    \centering
    \begin{tabular}{|l|l|l|l|l|l|l|l|l|l|l|l|l|l|}
\hline
M�s &  & 1� & 2� & 3� & 4� & 5� & 6� & 7� & 8� & 9� & 10� & 11� & 12� \\ \hline
\multirow{10}{*}{\begin{turn}{90} Etapas \end{turn}} & A & $\bullet$ & $\bullet$ &  &  &  &  &  &  &  &  &  &  \\ \cline{2-14} 
 & B & $\bullet$ & $\bullet$ & $\bullet$ &  &  &  &  &  &  &  &  &  \\ \cline{2-14} 
 & C &  &  & $\bullet$ & $\bullet$ &  &  &  &  &  &  &  &  \\ \cline{2-14} 
 & D &  &  &  & $\bullet$ &  &  &  &  &  &  &  &  \\ \cline{2-14} 
 & E &  &  &  &  & $\bullet$ & $\bullet$ &  &  &  &  &  &  \\ \cline{2-14} 
 & F &  &  &  &  & $\bullet$ & $\bullet$ &  &  &  &  &  &  \\ \cline{2-14} 
 & G &  &  &  &  &  &  & $\bullet$ & $\bullet$ &  &  &  &  \\ \cline{2-14} 
 & H &  &  &  &  &  &  &  & $\bullet$ & $\bullet$ &  &  &  \\ \cline{2-14} 
 & I &  &  &  &  &  &  &  &  &  & $\bullet$ & $\bullet$ &  \\ \cline{2-14} 
 & J &  &  &  &  &  &  &  &  &  &  & $\bullet$ & $\bullet$ \\ \hline
\end{tabular}
    \caption{Cronograma}
    \label{tab:my_label}
\end{table}



\bibliography{./bib/bibliography}


%
\chapter{INTRODU��O}\label{intro}

%Abertura e motiva��o...
Recentemente, o Brasil tem enfrentado uma s�rie de impactos imensur�veis ao meio ambiente e � sua biodiversidade, assim como danos sociais e econ�micos.
Exemplos recentes contemplam o rompimento de barragens de rejeito de min�rio, desmatamentos e inc�ndios florestais.


A regi�o do Pantanal Mato-grossense possui uma das maiores plan�cies de inunda��o do mundo, por�m, � caracterizada por um per�odo de estiagem muito forte entre os meses de junho a agosto. Nesse per�odo, os fazendeiros locais costumam atear fogo para ampliar as zonas de pasto e possibilitar o crescimento acelerado da produ��o pecu�ria \cite{AraujoSilva2015}. Atualmente, a regi�o do Pantanal, vive a maior seca desde a d�cada de 1970 que, atrelada com o efeito do uso do fogo, � respons�vel pelas maiores queimadas da hist�ria do local. Segundo \cite{MoraesEA2017}, esse tipo de fen�meno pode apresentar diversos efeitos ambientais negativos que v�o desde a perda da biodiversidade at� fortes impactos nos sistemas clim�ticos e terrestres.


%Uso do SR -- amplas areas e multitemporal
Eventos como estes motivam a elabora��o de estudos capazes de auxiliar no entendimento e monitoramento sistem�tico de �reas que implicam riscos � sociedade e ao meio ambiente.
Neste contexto, o sensoriamento remoto surge como uma tecnologia de extrema utilidade, uma vez que permite an�lises amplas no espa�o e de forma multitemporal.


%T�cnicas...
No �mbito do desenvolvimento de ferramentas de monitoramento, as t�cnicas de classifica��o e regress�o de dados \cite{Geron2019}, bem como uso de m�tricas espaciais \cite{herold2002} para caracteriza��o do dom�nio espacial, destacam-se como outra componente de grande import�ncia.
Quando aliadas aos dados obtidos por sensoriamento remoto, tais t�cnicas viabilizam a extra��o de conhecimento da superf�cie terrestre de forma automatizada.


%Ideia do que ser� feito...
Em face a esta problem�tica, este projeto de pesquisa prop�e analisar as din�micas de uso e cobertura do solo em uma regi�o do pantanal e, por meio de um modelo de regress�o, verificar rela��es desempenhadas por tais din�micas sobre a ocorr�ncia, ou n�o, de inc�ndios. O modelo de regress�o mencionado surge como ferramenta para o mapeamento de �reas suscet�veis a inc�ndios.


%Organiza��o do texto
O texto a seguir est� organizado da seguinte forma: na Se��o~\ref{Obj} � enunciado o objetivo geral e listados os objetivos espec�ficos do projeto; justificativas e relev�ncias desta pesquisa s�o expostas na Se��o~\ref{Jus}; uma breve discuss�o sobre m�todos de classifica��o e regress�o de dados bem como sobre m�tricas espaciais � apresentada na Se��o~\ref{Rev}; a Se��o~\ref{Proposta} � reservada para exposi��o e discuss�o da proposta de pesquisa em si; na Se��o~\ref{MatMet} � apresentada uma descri��o da �rea de estudo e dos dados dispon�veis, assim como sobre os procedimento metodol�gicos a serem realizados; por fim, � delineada uma concep��o de como a pesquisa proposta dever� ser desenvolvida, seguido de um cronograma de atividade, exposto na Se��o~\ref{Cron}.



%
%\chapter{OBJETIVOS}\label{Obj}

Este projeto de pesquisa tem como objetivo principal investigar as din�micas de mudan�a no uso e cobertura do solo em de regi�es inseridas no Pantanal Mato-grossense, no per�odo de 2000 a 2019, e que posteriormente foram afetadas pelos inc�ndios recentes ocorridos no ano de 2020. 
Tal investiga��o ser� fomentada por dados multitemporais obtidos por sensoriamento remoto e por t�cnicas de Processamento Digital de Imagens e Aprendizado de M�quina.


Como objetivos espec�ficos, s�o delineados:

{%\singlespace
\begin{enumerate}%[i --]
\item Implementa��o de procedimento para caracteriza��o das din�micas temporais sobre o uso e cobertura do solo;

\item Aplica��o do procedimento implementado sobre uma s�rie hist�rica de imagens Landsat compreendida entre os anos de 2000 e 2019 em uma �rea de estudo espec�fica;

\item An�lise e caracteriza��o das din�micas do comportamento das �reas afetadas ou n�o pelos inc�ndios recentes;

\item Publica��o dos resultados obtidos em eventos ou peri�dicos cient�ficos.
\end{enumerate}
}    %ok!
%
%\chapter{JUSTIFICATIVA E RELEV�NCIA DO PROJETO} \label{Jus}


O projeto tem relev�ncia nas �reas ambiental e computacional. No ponto de vista ambiental, o projeto apresentar� uma investiga��o multitemporal sobre as din�micas no uso e cobertura do solo e os inc�ndios recentes na regi�o do Pantanal Mato-grossense. Com isso, vislumbra-se identificar rela��es entre as itera��es antr�picas e o meio atingido pelos inc�ndios.

Em rela��o ao vi�s computacional, este projeto visa desenvolver uma metodologia que converte uma s�rie temporal de imagens de sensoriamento remoto em um descritor sobre as itera��es no uso e cobertura do solo, que por sua vez torna-se �til na constru��o de modelo de previs�o/regress�o capaz de avaliar a suscetibilidade de outras regi�es vizinhas � inc�ndios.


 %ok!
%
%\chapter{FUNDAMENTA��O TE�RICA}\label{Rev}


%%%%%%%%%%%%%%%%%%%%%%%%%%%%%%%%%%%%%%%%%%%%%%%%%%%%%
\section{Classifica��o e regress�o}\label{revClassRegress}

Estudos em Sensoriamento Remoto com foco na classifica��o de imagens tem atra�do a aten��o da comunidade cient�fica, pois os resultados provenientes destas t�cnicas servem de base para diversas aplica��es ambientais e socioecon�micas. 
%Por�m, a classifica��o de imagens de Sensoriamento Remoto, de forma a criar um mapa tem�tico � um desafio, uma vez que a complexidade da paisagem, a escolha da imagem, o processamento e abordagem de classifica��o podem afetar o sucesso do procedimento. 
Sua import�ncia t�m motivado ainda o desenvolvimento de novas propostas capazes de proporcionar resultados de classifica��o cada vez mais acurados \cite{lu2007}.

Um classificador � representado por uma fun��o $F: \mathcal{X} \rightarrow \mathcal{Y}$ , que associa elementos do espa�o de atributos $\mathcal{X}$ a uma das classes de $\Omega = \left\{ \omega_{1}, \omega_{2}, \ldots, \omega_{c}\right\}$, com $c \in \mathbb{N}^{*}$, a partir de um dado r�tulo (indicador) de classe em $\mathcal{Y} = \left\{ 1,2,\ldots,c\right\}$. Nestas condi��es, para $\textbf{x} \in \mathcal{X}$ e $y \in \mathcal{Y}$, $y = F(\textbf{x})$ indica que $\textbf{x}$ pertence � classe $\omega_{y}$.

A classifica��o de imagens consiste na aplica��o de $F$ sobre os vetores de atributos dos pixels (padr�es) que comp�em uma imagem $\mathcal{I}$, definida sobre um reticulado (suporte) $\mathcal{S} \subset \mathbb{N}^{2}$, cujo resultado de classifica��o pode ser denotado por $F(\mathcal{I})$. Com rela��o � imagem em que � conduzido o processo de classifica��o, $\mathcal{I}(s) = \mathbf{x}$, denota que o pixel $s \in \mathcal{S}$ de $\mathcal{I}$ possui atributos representados pelo vetor $\mathbf{x} \in \mathcal{X}$, ainda $C(s) = \omega_{y}$ tal que $F(\textbf{x})=y$.

Os diferentes m�todos de classifica��o de imagem propostos na literatura podem ser entendidos como maneiras distintas de modelar a fun��o $F: \mathcal{X} \rightarrow \mathcal{Y}$ e aplic�-la na classifica��o de $\mathcal{I}$. Os paradigmas supervisionado e n�o supervisionado s�o duas formas/abordagens usualmente adotadas pelos m�todos de classifica��o de imagens. No aprendizado supervisionado s�o fornecidos ao m�todo um conjunto de exemplos de treinamento $\mathcal{D} = \left\{ (\mathbf{x}_{i}, y_{i})\in \mathcal{X} \times \mathcal{Y} : i = 1,2,\ldots,m \right\}$ composto por $m \in \mathbb{N}^{*}$ vetores, onde seus respectivos r�tulos da classe associada s�o conhecidos. Ent�o, o mapeamento entre $\mathcal{X}$ e $\mathcal{Y}$ definido por $F$ representa o conhecimento adquirido das informa��es observadas em $\mathcal{D}$. %No aprendizado n�o supervisionado, os dados s�o analisados pelo algoritmo, a fim de definir agrupamentos sobre os mesmos \cite{monard2003}.
M�todos como M�quina de Vetores Suporte e Florestas Aleat�rias t�m mostrado efici�ncia na classifica��o de imagens de sensoriamento remoto \cite{LiEA2014}.



Em analogia aos m�todos de classifica��o, um modelo de regress�o abrange a
obten��o de uma fun��o $G:\mathcal{X} \rightarrow \mathbb{R}$, que uma vez modelada atrav�s de um conjunto de observa��es $\mathcal{D} = \left\{ (\mathbf{x}_{i}, y_{i})\in \mathcal{X} \times \mathbb{R} : i = 1,2,\ldots,m \right\}$, permite extrapolar valores reais face a um dado vetor de atributos.
Dentre diversos modelos existentes na literatura, a Regress�o Log�stica, tem sido amplamente empregada na estima��o da probabilidade de pertin�ncia ``elemento-classes'' em casos de associa��o bin�ria (i.e., envolvendo apenas duas classes) \cite{Geron2019}, o qual pode ser entendido como uma composi��o entre as fun��o linear e sigmoide, ou seja:
\begin{equation}\label{eqC9ModLogit}
g(\mathbf{x};\mathbf{\theta}) = \frac{1}{1+e^{-\left( \mathbf{\theta}^T \mathbf{x} \right)}}.
\end{equation}



Devido a sua estrutura sigmoide, possui imagem definida sobre o intervalo $[0,1]$. No contexto dos problemas de classifica��o\footnote{As classes mencionadas podem ser interpretadas como ``aus�ncia'' ou ``presen�a'' de determinada caracter�stica.}, $g(\mathbf{x}_i;\mathbf{\theta}) < 0,5$ implica que $\mathbf{x}_i$ deve pertencer � classe de indicador $0$; analogamente, dever� estar associado � classe de indicador $1$ quando $g(\mathbf{x}_i;\mathbf{\theta}) \geq 0,5$. Ainda, � poss�vel interpretar os resultados gerados como uma ``probabilidade de pertin�ncia'' que migra gradativamente entre duas condi��es opostas.


%%%%%%%%%%%%%%%%%%%%%%%%%%%%%%%%%%%%%%%%%%%%%%%%%%%%%
\section{M�tricas espaciais}\label{revMetricas}

M�tricas espaciais s�o medidas derivadas da an�lise digital de mapas a fim de quantificar a heterogeneidade espacial em uma determinada escala e resolu��o \cite{herold2002}. Tais medidas permitem caracteriza��es quantitativas sobre a composi��o espacial, configura��es do \textit{habitat} e dos tipos de cobertura do solo. %A combina��o do Sensoriamento Remoto com m�tricas espaciais podem ser �teis na obten��o de informa��es consistentes e detalhadas sobre a estrutura urbana, possibilitando assim maior representa��o e melhor entendimento do processo de crescimento urbano \cite{deng2009}.
%Como instrumentos de an�lise sobre a constitui��o da paisagem, v�m-se adotando as m�tricas espaciais como mecanismos de compreens�o das implica��es do crescimento urbano. 
Por meio destas medidas, � poss�vel identificar estruturas, padr�es e conectividade de �reas, e ainda, o n�mero, tamanho, forma e proximidade das manchas de uso e cobertura do solo.
%, avaliando os impactos de poss�veis expans�es de cidades \cite{lee2011}.
%Nesta pesquisa � proposto o uso das m�tricas espaciais denominadas por porcentagem de cobertura, coeficiente de varia��o das �reas das manchas, densidade de manchas e densidade de bordas.
Quatro tipos de m�tricas espaciais mostram-se �teis na caracteriza��o do espa�o, s�o elas: porcentagem de cobertura, coeficiente de varia��o das �reas das manchas, densidade de manchas e densidade de bordas.

Inicialmente, para a defini��o de mancha � observada uma vizinhan�a espacial, definida por:
\begin{equation}\label{vizinhos}
\mathcal{V}_{\rho}\left(s_{i}\right) = \left\{ s_{i} \in \mathcal{S} : d\left( s_{i},t\right) < \rho; t \in \mathcal{S} \right\},
\end{equation}
sendo $d(\cdot ,\cdot)$ a dist�ncia do m�ximo, isto �: $d\left( a, b \right) = \max\left\{ \left| a_{1} - b_{1} \right|, \left| a_{2} - b_{2} \right| \right\}$ para $a = \left\{ a_{1}, a_{2} \right\}$ e $b = \left\{ b_{1}, b_{2} \right\}$ elementos de $\mathcal{S}$. Ainda, $\rho$ representa o raio de vizinhan�a de $s_{i}$.

Uma vez definida a vizinhan�a de $s_{i}$, s�o identificadas como manchas cada conjunto de posi��es que apresentam classe comum e s�o espacialmente conectadas segundo uma vizinhan�a de ordem 1 (\textit{i.e.}, vizinhan�a-4). Formalmente, uma mancha $M^{(y)}_{j}\left( s_{i}, \rho \right)$ � o conjunto representado por:

\begin{equation}\label{manchas}
M^{(y)}_{j}\left( s_{i}, \rho \right) = \left\{ t \in \mathcal{V}_{\rho}(s_{i}) : C(t) = \omega_{y}, C(t)=C(r), \left\| t-r \right\| \leq 1 \right\}.
\end{equation}

A porcentagem de cobertura que representa a propor��o de cada classe sobre a �rea total, � expressa por:
\begin{equation}\label{pctcbrtr}
P_{y} = \frac{A_{y}}{A},
\end{equation}
onde $A_{y} = \#\bigcup\limits^{m_{y}}_{j=1}M^{(y)}_{j}\left( s_{i}, \rho \right)$ representa a �rea das manchas associadas � classe $\omega_{y}$, dada pelo total de pixels associados a esta classe, e $A = \#\bigcup\limits^{c}_{k=1} \bigcup\limits^{m_{k}}_{j=1}M^{(k)}_{j}\left( s_{i}, \rho \right)$ determina a soma das �reas de todas as manchas. Ainda, $m_{k}$ representa o n�mero de manchas relacionadas a uma dada classe $\omega_{k} \in \Omega$.

O coeficiente de varia��o das manchas, por ser expresso em porcentagem, permite comparar a varia��o das �reas das manchas entre diferentes classes. Tal coeficiente � expresso por:
\begin{equation}\label{coefvar}
CV_{y} = \frac{\sigma\left( M^{(y)}_{j}\left( s_{i}, \rho \right) \right)}{\mu\left( M^{(y)}_{j}\left( s_{i}, \rho \right) \right)}; \ j=1, \ 2, \ \dots, \ m_{y} \ ,
\end{equation}
onde $CV_{y}$, $\sigma\left( M^{(y)}_{j}\left( s_{i}, \rho \right) \right)$ e $\mu\left( M^{(y)}_{j}\left( s_{i}, \rho \right) \right)$ s�o, respectivamente, o coeficiente de varia��o, desvio padr�o e m�dia das �reas das manchas associadas � classe $\omega_{y}$ referente a vizinhan�a $\mathcal{V}_{\rho}\left(s_{i}\right)$.

A densidade de manchas quantifica a propor��o do n�mero de manchas de determinada classe em rela��o � �rea de todas as manchas. Tal m�trica � dada por:
\begin{equation}\label{denman}
D_{y} = \frac{m_{y}}{A},
\end{equation}

Por fim, densidade de bordas quantifica a propor��o do comprimento das bordas em rela��o � �rea de todas as manchas, dado por:
\begin{equation}\label{denbor}
B_{y} = \frac{\sum\limits^{m_{y}}_{j=1}b^{(y)}_{j}\left( s_{i}, \rho \right)}{A},
\end{equation}
onde $b^{(y)}_{j}$ corresponde ao per�metro de uma mancha $M^{(y)}_{j}\left( s_{i}, \rho \right)$.



%%%%%%%%%%%%%%%%%%%%%%%%%%%%%%%%%%%%%%%%%%%%%%%%%%%%%
\section{Breve discuss�o sobre Florestas Aleat�rias}\label{revRF}


Florestas Aleat�rias ({\itshape Random Forest} -- RF) \cite{Breiman2001} compreende uma abordagem de classifica��o que combina um conjunto de �rvores de decis�o. Dois pontos importantes desta abordagem referem-se ao uso da amostragem {\itshape bootstrap} na determina��o de conjuntos de dados de treinamento e da vota��o por maioria para a tomada de decis�o final.

Diante do fato das �rvores de decis�o serem sens�veis aos dados de treinamento, os modelos obtidos tendem a apresentar certo grau de distin��o entre si. Como consequ�ncia, durante o processo de classifica��o uma �rvore dever� proteger outra �rvore do seu pr�prio erro, logo, a classifica��o ocorre de forma err�nea somente quando as �rvores erram em favor de uma mesma classe.

Outro fator que aumenta ainda mais a distin��o entre as �rvores � a sele��o aleat�ria de apenas parte dos atributos dispon�veis para uso em cada uma das �rvore que comp�e a combina��o. 
Uma vez que a combina��o das decis�es � tomada com base na vota��o por maioria, espera-se que uma grande quantidade de modelos parcialmente distintos deve superar, em opini�o, um �nico modelo constituinte.

%As caracter�sticas mencionadas torna o RF um dos m�todos de Aprendizado de M�quina mais robustos dentro desta �rea de pesquisa.

%
%%\hypertarget{estilo:capitulo}{}
%%%%%%%%%%%%%%%%%%%%%%%%%%%%%%%%%%%%%%%%%%%%%%%%%
\chapter{MODELAGEM DA SUSCETIBILIDADE � INC�NDIOS} \label{Proposta}

O diagrama ilustrado na Figura~\ref{framework} cont�m o racional desta proposta de pesquisa. Seus diferentes constituintes compreendem os conceitos discutidos na Se��o~\ref{Rev}.

%%%%%%%%%%%%%%%%%%%%%%%%%%%%%%
\begin{figure}[h!]
\centering
\includegraphics[width=\textwidth]{./figs/FrameworkProjFelipeNascimento.pdf}
\caption{Racional da proposta de pesquisa.}\label{framework}
\end{figure}

Partindo de uma s�rie multitemporal de imagens obtidas por sensoriamento remoto, especificamente compreendida pelos anos de 2000, 2005, 2010, 2015 e 2019\footnote{Tais anos est�o pass�veis de modifica��o perante situa��es extremas, por exemplo, a cobertura de nuvem ou aus�ncia de dados.}. Sobre cada uma destas imagens � efetuada a sele��o de amostras de uso e cobertura do solo dentre quatro classes poss�veis: alta biomassa; m�dia biomassa; baixa biomassa; outros. Ao passo que a classe de baixa bomassa corresponde a �reas de solo exposto ou agropastoril, as classes de alta e m�dia biomasssa correspondem a locais de floresta e regenera��o recente. A quarta classe, denominada ``outros'' � destinada a corpos d'�gua e demais alvos cujo ac�mulo de biomassa � improv�vel.

Posteriormente, cada uma das imagens da s�rie multitemporal, devidamente acompanhada das amostras de uso e cobertura do solo, � ent�o submetida a um processo de classifica��o com uso do m�todo RF (Se��o~\ref{revRF}). A parametriza��o e avalia��o dos resultados de classifica��o � guiado por procedimento {\itshape grid-search} e avalia��es via coeficiente kappa \cite{CongaltonGreen2009}.

Em posse dos resultados de classifica��o, ser�o aplicadas m�tricas espaciais para extra��o e caracteriza��o das din�micas locais em um espa�o de atributos de dimens�o elevada. Ao admitir as quatro m�tricas\footnote{Supondo um valor fixo para o raio de vizinhan�a ($\rho$)} discutidas na Se��o~\ref{revMetricas} (Equa��es~\ref{pctcbrtr} a \ref{denbor}) e que a s�rie multitemporal possui quatro instantes (i.e., 2000, 2005, 2010, 2015 e 2019), � proporcionado ao fim uma imagem cujos pixels est�o associados a um vetor de dimens�o 20.

Por sua vez, com referencia em uma imagem recente da �rea de estudo, � efetuada a coleta de amostras sobre regi�es afetadas, ou n�o, por inc�ndios. Este processo leva a composi��o de um conjunto rotulado das din�micas locais. Com uso deste conjunto � viabilizada a determina��o de um modelo de regress�o log�stica, que por sua vez pode ser empregado na extrapola��o do comportamento ``afetadas'' e ``n�o afetadas'' como fun��o das din�micas locais.

O modelo de regress�o obtido, quando aplicado sobre �reas n�o rotuladas, ou mesmo sobre regi�es vizinhas da �rea de estudo que tenham sido submetidas ao mesmo processo de classifica��o e extra��o de m�tricas espaciais, deve proporcionar uma infer�ncia a respeito da suscetibilidade a inc�ndios. A avalia��o deste processo pode ser feito mediante a compara��o entre a estima��o gerada pelo modelo obtido em locais onde � conhecida a ocorr�ncia, ou n�o, de inc�ndios. Mais uma vez, o coeficiente kappa deve ser empregado neste prop�sito de avalia��o.

     %fazer
%
%%\hypertarget{estilo:capitulo}{}

\chapter{RESULTS AND DISCUSSION} \label{MatMet}

The tweets filtered based on the words presented in the \ref{refmeto} methodology, presented a large part of the tweets related to the context of rain and flooding. The use of other words such as 'raio' and 'tempestade', result in tweets using these words to designate a metaphorical context or with a sense displaced from the phenomena of rain and flooding. Therefore, the application of the word 'chuva' and its variations present a more stable meaning to refer to meteorological phenomena.

However, it can be observed that the filtering algorithm detects tweets using the word rain in a more poetic sense (Table \ref{tweets_filt}).

\begin{table}[H] \label{tweets_filt}
	\caption{Related and unrelated Tweets}
	\begin{tabular}{ll}
		\hline
		\multicolumn{2}{c}{Examples of Tweets}                                                                                                                                                                                                                                                                                             \\ \hline
		\multicolumn{1}{c|}{Related tweets}                                                                                                                                          & \multicolumn{1}{c}{Tweets out of context}                                                                                                           \\ \hline
		\multicolumn{1}{l|}{\begin{tabular}[c]{@{}l@{}}quem aqui gosta de pokemon?\textbackslash{}nvideo\\  de dias atras  porque a chuva estragou \\ meus planos hoje\end{tabular}} & \begin{tabular}[c]{@{}l@{}}minha força esta na solidao. não tenho\\ medo nem de chuvas tempestivas \\ nem de grandes ventanias soltas.\end{tabular} \\ \hline
		\multicolumn{1}{l|}{\begin{tabular}[c]{@{}l@{}}sabado com chuvas e minhas aluna vieram\\ fazer um alongamento para tira toda preguica\end{tabular}}                          & \begin{tabular}[c]{@{}l@{}}a ordem e seguir em frente romper a \\ tempestade e não se ater aos ventos, raios \\ e chuvas.\end{tabular}              \\ \hline
	\end{tabular}
\end{table}

%%%%%%%%%%%%%%%%%%%%%%%%%%%%%%%%%%%%%%%%%%%%%%%%%
% The data were processed for the same time window and an exploratory data analysis was performed using the graph below (\ref{fig:graph}). The pluviometric data were collected from the 833A pluviometer, in which the accumulated rain level per day was added. This rain gauge measures the rain level  every 10 minutes. Radar precipitation data was collected in a cell that covers the same region as the pluviometer, thus processing the data for the analyzed temporal window. Tweets were collected within a radius of \si{2000 \meter} and filtered based on a list of words. The words were: 'chuva', 'chove', 'chuvoso', 'chuvosa', according to \cite{de2021effect}, these words are less spatially and temporally volatile than more local and idiosyncratic terms specifically related to the city of São Paulo (e.g. 'garoa' and 'tempestade'). 

After the unification of all attributes in a single DataFrame, a graph was plotted relating these variables to the time window studied \ref{fig:graph}.

\begin{figure}[H]
	\centering
	\includegraphics[width=1.2\textwidth]{figs/newplot.png}
	\caption{Plot}
	\label{fig:graph}
\end{figure}

The plotting of data indicates that on every day that flooding occurred there was an incidence of rain indicated by the rain gauge and radar and rain-related tweets. As can be seen in the figure \ref{fig:graph}, the frequency of flooding on a given day determines the size of the black circle, and that the days with the highest incidence of flooding were not peaks of rain detected by meteorological equipment.

To determine the statistical relationships a series of tests were performed. First, it was necessary to determine if the attributes were a normal distribution, so that in this way parametric or non-parametric statistical tests could be applied. For this verification, the Shapiro Wilk test was used (\ref{shapiro}).

\begin{table}[H]\label{shapiro}
	\caption{P-Values results}
	\begin{center}
	\begin{tabular}{ll}
		\hline
		\multicolumn{2}{c}{Shapiro-Wilk test $\alpha=0.05$}                              \\ \hline
		\multicolumn{1}{c|}{Attributes}      & \multicolumn{1}{c}{P-value} \\ \hline
		\multicolumn{1}{l|}{Rain Gauge}      & 7.175038015984347e-13       \\ \hline
		\multicolumn{1}{l|}{Tweet Frequence} & 2.0394061550632614e-07      \\ \hline
		\multicolumn{1}{l|}{Radar}           & 7.194410709808908e-15       \\ \hline
		\multicolumn{1}{l|}{Flood}           & 1.445436313501046e-14       \\ \hline
	\end{tabular}
\end{center}
\end{table}

In the table above \ref{shapiro}, the p-values of all attributes were below the limit of $\alpha$, discarding the null hypothesis that the distribution is normal. Based on the results, to calculate the correlations, Spearman's non-parametric test (\ref{fig:corr}) was used for these data. 

\begin{figure}[H]
	\centering
	\includegraphics[width=0.47\textwidth]{figs/corr.png}
	\caption{Spearman Correlation}
	\label{fig:corr}
\end{figure}

From the correlations, it is clear that the radar (p3), followed by the tweets, have the highest correlation with the frequency of flooding and also the most significant value. According to \citen{statistics_solutions_2021}, the cited attributes indicates a strong correlation with floods. 

Furthermore, rainfall data indicate the occurrence of precipitation in opposition to Radar. As seen in figure \ref{fig:graph}, the graph indicates the low correlation between these two attributes.

From the results, the most relevant correlations will also exert greater influence on the flooding prediction.
      %fazer
%
%\chapter{CONCLUSION AND PERSPECTIVES} \label{Cron}

The first project work indicates a statistical relation between the tweets associated with the meteorological and hydrological context with flooding. Due to the myriad of factors linked to the outbreak of this hydrological phenomenon, other supplementary monitoring mechanisms are needed. In short, Twitter can be a powerful complementary tool. By joining this social network with other monitoring stations, sophisticated monitoring and alerting systems can be developed.

In general, in the second project it can be observed that the value of the attributes with the floods have moderate and strong correlations. Demonstrating that there is potential to submit data to classification algorithms.

In possession of primary analysis, the next step is to prepare the data for Machine Learning algorithms, optimize data quality and detect outliers. 


%\newpage{}
  %quase...
%
%%\newpage
%
%\bibliography{./bib/bibNascimento__}


\inicioIndice
\end{document}