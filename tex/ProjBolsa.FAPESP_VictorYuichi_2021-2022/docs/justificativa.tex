\chapter{JUSTIFICATIVA E RELEV�NCIA DO PROJETO} \label{Jus}


O projeto tem relev�ncia nas �reas ambiental e computacional. No ponto de vista ambiental, o projeto apresentar� uma investiga��o multitemporal sobre as din�micas no uso e cobertura do solo e os inc�ndios recentes na regi�o do Pantanal Mato-grossense. Com isso, vislumbra-se identificar rela��es entre as itera��es antr�picas e o meio atingido pelos inc�ndios.

Em rela��o ao vi�s computacional, este projeto visa desenvolver uma metodologia que converte uma s�rie temporal de imagens de sensoriamento remoto em um descritor sobre as itera��es no uso e cobertura do solo, que por sua vez torna-se �til na constru��o de modelo de previs�o/regress�o capaz de avaliar a suscetibilidade de outras regi�es vizinhas � inc�ndios.


